\documentclass{article}
\usepackage[margin=0.75in]{geometry}
\usepackage{amsmath}
\usepackage{physics}
\usepackage{listings}
\usepackage{caption}
\usepackage{graphicx}
\usepackage{algorithm}
\usepackage{algpseudocode}
\usepackage{setspace} \doublespacing
\graphicspath{{Images/}}


\title{PHYS 471 Research Plan: Autocatalytic Reaction Networks}
\author{Varun Varanasi}
\begin{document}


\maketitle

My proposed project for PHYS 471 is an investigation of autocatalytic reaction networks under the guidance of Professor Jun Korenaga. This project is intended to serve as the first semester of a two semester senior thesis for my degree in Intensive Physics. The scope of this project lies under the large umbrella of theoretical biology studying the origin of life. 

In particular, my project is focused on studying the structure and dynamics of autocatalytic reaction networks. These reaction networks are defined by their ability to produce species that catalyze other steps within the reaction network. In our contemporary understanding of abiogenesis, the origin of life, autocatalysis and self-replication are necessary steps in the evolution of complex and interlinked biological systems. Recent advances in autocatalytic reaction theory have developed an algorithm to identify RAF reaction networks. This subtype of reaction networks can be understood as the barebones of an autocatalytic reaction network in which each reaction is catalyzed by a product of another reaction in the reaction set. Furthermore, each product produced in the network is a function of the action of a series of reactions on an initial food set (starting materials). Despite these models remaining our best understanding of autocatalytic reaction networks, they remain highly idealized. 

The goal of this project is broadly to test the robustness of the RAF model. The project scope will rely heavily on theory and as such a majority of this semester's work will be focused on gaining the necessary background to approach this project. In particular, the first few tasks outlined for the semester are to conduct a literature review on the state of research in autocatalytic reaction networks, a reproduction of key results from seminal papers in the field, and a reading of a textbook focused on complexity and criticality. With this understanding, the remainder of this project, likely bleeding into next semester, will be to relax RAF model assumptions and develop more practical models of abiogenesis. 

\end{document}